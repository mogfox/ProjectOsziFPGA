\subsection{Trigger}\label{trigger}
\subsubsection{Projektteilbereich Übersicht}
Der Edge-Trigger ist wesentlich für die Unterteilung in Pakete. Dieser reagiert auf eine vorgegebene, konfigurierbare Schwellspannung in jedem der drei Modi. Der gewünschte Modus wird vom User-Interface vorgegeben. Einen sogenannten "Single-Shot" kennt diese Komponente nicht, da diese ausschließlich primitiv triggert und sich nicht um "höhere Angelegenheiten" kümmert.
\subsubsection{Funktion}
\paragraph{Modus: Rising Edge}
Bei der \textit{Rising-Edge-Detection} löst der Trigger beim Erreichen und Überschreiten des Schwellwertes (in der Grafik: \textcolor{red}{rot}) aus. Es muss jedoch immer davor, aufgrund der Hysterese, ein vom Schwellwert abhängiger unterer Wert (in der Grafik: \textcolor{blue}{blau}) erreicht werden (arming\_level\_low). Damit wird falsches, zu häufiges Auslösen wegen höherfrequenten und/oder überlagerten Signalen vermieden. Signale mit einer zu kleinen Amplitude ($\leq$ Hysterese) können nicht getriggert werden. In der Grafik \ref{risingEdge} sind 3 Szenarien Abgebildet, nur im ersten löst der Trigger aus.
\begin{figure}[h]
	\begin{center}
		\includegraphics[width=15cm]{SAUER/Grafiken/Trigger/TriggerEdgeRising.jpg}
		\caption{Trigger Rising-Edge-Detection}
		\label{risingEdge}
	\end{center}
\end{figure}
\paragraph{Modus: Falling Edge}
Bei der \textit{Falling-Edge-Detection} löst der Trigger beim Erreichen oder Unterschreiten des Schwellwertes (in der Grafik: \textcolor{red}{rot}) aus. Es muss jedoch immer davor, aufgrund der Hysterese, ein vom Schwellwert abhängiger oberer Wert (in der Grafik: \textcolor{blue}{blau}) erreicht werden (arming\_level\_high). Damit wird falsches, zu häufiges Auslösen wegen höherfrequenten und/oder überlagerten Signalen vermieden. Signale mit einer zu kleinen Amplitude ($\leq$ Hysterese) können nicht getriggert werden. In der Grafik \ref{fallingEdge} sind 3 Szenarien Abgebildet, nur im ersten löst der Trigger aus.
\begin{figure}[h]
	\begin{center}
		\includegraphics[width=15cm]{SAUER/Grafiken/Trigger/TriggerEdgeFalling.jpg}
		\caption{Trigger Falling-Edge-Detection}
		\label{fallingEdge}
	\end{center}
\end{figure}
\paragraph{Modus: Any Edge}
Bei der \textit{Any-Edge-Detection} löst der Trigger sowohl beim Unterschreiten als auch beim Überschreiten des Schwellwertes (in der Grafik: \textcolor{red}{rot}) aus. Es muss jedoch immer davor, aufgrund der Hysterese, ein vom Schwellwert abhängiger oberer oder unterer Wert (in der Grafik: \textcolor{blue}{blau}) überschritten bzw. unterschritten werden (arming\_level\_high, arming\_level\_low). Damit wird falsches, zu häufiges Auslösen wegen höherfrequenten und/oder überlagerten Signalen vermieden. Signale mit einer zu kleinen Amplitude ($\leq$ Hysterese) können nicht getriggert werden. In der Grafik \ref{anyEdge} sind 4 Szenarien Abgebildet, in der ersten Zeile funktioniert das Triggern, in der zweiten nicht.
\begin{figure}[h]
	\begin{center}
		\includegraphics[width=15cm]{SAUER/Grafiken/Trigger/TriggerEdgeAny.jpg}
		\caption{Trigger Any-Edge-Detection}
		\label{anyEdge}
	\end{center}
\end{figure}
\paragraph{Schwellwert, Hysterese und Modi}
Diese Konfigurationseinstellungen werden vom User-Interface vorgegeben. (siehe: \ref{Protokoll}) Für den Schwellwert kann jeder Wert ($\approx$ Spannung) eingestellt werden, jedoch sollte darauf geachtet werden, dass die Hysterese im aktuellen Modus den Wertebereich nicht verlässt. Die Trigger-Unit begrenzt sicherheitshalber die Hysterese dynamisch so, dass diese den Wertebereich nicht verlassen kann.($arming\_level\_low \geq 0, arming\_level\_high \leq 4095$) Der Schwellwert wird nicht in Volt angegeben, sondern im Bereich der Rohdaten des ADCs (0 - 4095).
Eine Hysterese im Bereich von 20 bis 50 digits wird empfohlen.
\subsubsection{VHDL-Design-Unit}
Die Komponente ist sehr einfach gehalten und auch zu bedienen bzw. zu implementieren. 
\begin{figure}[h]
	\begin{center}
		\includegraphics[width=15cm]{SAUER/Grafiken/Trigger/EntityTrigger.png}
		\caption{Entity Trigger}
	\end{center}
\end{figure}
\begin{tabular}[h]{|l|l|l|}
	\hline
	Port & Typ & I/0\\
	\hline\hline
	RESET\_n & std\_logic & in\\
	\hline
	CLK & std\_logic & in\\
	\hline
	values\_in\_i & natural & in\\
	\hline
	trigger\_threshold\_i & natural & in\\
	\hline
	trigger\_hyst\_i & natural & in\\
	\hline
	trigger\_mode & trigger\_types & in\\
	\hline
	trigger & std\_logic & out\\
	\hline
\end{tabular}
\subsubsection{Design-Unit Test}