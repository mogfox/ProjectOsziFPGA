\documentclass{scrartcl}

\usepackage[utf8]{inputenc}
\usepackage[T1]{fontenc}
\usepackage{lmodern}
\usepackage[ngerman]{babel}
\usepackage{amsmath}

\usepackage{color}
\usepackage{graphicx}
\usepackage{hyperref}
\usepackage{tabularx}
\begin{document}

\title{Pflichtenheft FPGA-Oszilloskop}
\author{Radike, Sauer, Wolf}
\date{\today}
\maketitle
\begin{tabular}[h]{l|l|l|l} %l = left c = center r = right
		Version & Datum & Autor & Änderungsgrund/Bemerkungen\\
		\hline
		1.0 & 17.3.2021 & Sauer Lukas & Erstellung
\end{tabular}
\tableofcontents 
\newpage

\section{Übersicht}
In diesem Dokument wird die Umsetztung der im Pflichtenheft angeführten Anforderungen beschrieben und ganau spezifiziert. 
\subsection{Projektbeschreibung}
Im Projekt Oszi wird ein Oszilloskop mittels eines FPGAs zu realisieren. Das Oszilloskop soll ein Frequenzband von 0Hz bis 1MHz einlesen und einen Spannungsbereich von -50V bis +50V abdecken. Die Spannungskurfe soll über ein Computerprogramm graphisch ausgegeben werden. Über das Programm soll ebenfalls der Spannungs- und Zeitbereich der Ausgabe einzustellen sein. Das Oszilloskop solle triggerbar sein und Tastköpfe kalibrieren können. Das Projekt besteht aus drei Aufgabenbereichen, dem Analog-Front-End, der Datenverarbeitung mit dem FPGA und der Ausgabe am PC.

\subsection{Aufteilung}
Das Projekt wird in drei, parallel laufende Teilbereiche geteilt. Diese sind voneinader größtenteils unabhängig, sind nur über eine Schnittstelle miteinander verbunden und werden von jewils einem Projektmitglied hauptsächlich bearbeitet:
\begin{itemize}
\item analog front end (siehe: \ref{wolf})
\item digital data processing (siehe: \ref{sauer})
\item user interface (siehe: \ref{radike})
\end {itemize}


\section{digital signal processing} \label{sauer}
\textbf{Zuständig für diesen Bereich: Sauer Lukas}\\
\subsection{Vorwort}
Dieser Teilbereich ist für das Einlesen, die temporiäre Speicherung, die digitale Verarbeitung und für das Weilterleiten der digitalen Messwerte zum user interface (siehe: \ref{radike}) verantwortlich. Die technisch schwieriger zu realisierenden Anforderungen an das System sind das digitale Filtern, die Kompensierung des Amplitudengangs sowie das triggern und Zusammenfassen in Pakete, welche zum user interface geschickt werden.\\
In diesem Abschnitt des Projekts sind die Hardwarebeschreibungssprache VHDL und Kenntnisse in digitaler Signalverarbeitung besonders gefragt. Es wird sich hauptsächlich mit dem Entwicklungsboard "DE10-Lite" beschäftigt mit der der Software "Quartus Prime" und "ModelSim".

\subsection{Quartus-Prime Projekterstellung}

\begin{tabular}[h]{|l|l|} %l = left c = center r = right
	\hline
	status & abgeschlossen\\
	\hline
	beanspruchte Zeit & 4 h\\
	\hline
	erledigt bis & / \\
	\hline
\end{tabular}
Das Quartus-Prime Projekt wird an das DE10-Lite-Board abgestimmt und erste Testversuche werden unternommen. Der Code für die Hardwarebeschreibung wird auf mehrere Files aufgeteilt, verknüpft und in einer Top-Level-Design-Unit zusammengeführt. 

\subsection{Interner ADC}

\begin{tabular}[h]{|l|l|} %l = left c = center r = right
	\hline
	status & abgeschlossen \& fehlgeschlagen\\
	\hline
	beanspruchte Zeit & 16 h\\
	\hline
	erledigt bis & 4.03.2021 \\
	\hline
\end{tabular}
Als erster Schritt wird der Interne ADC verwendet, um einen Datenstream herinzubekommen und erste Versuche für die weiter Datenverarbeitung auch ohne externen ADC zu ermöglichen. --> Fehlgeschlagen

\subsection{externer ADC}

\begin{tabular}[h]{|l|l|} %l = left c = center r = right
	\hline
	status & abgeschlossen\\
	\hline
	beanspruchte Zeit & 4 h\\
	\hline
	erledigt bis & 4.03.2021 \\
	\hline
\end{tabular}
Auswahl des externen ADCs. Absprache mit dem Zuständigen für das analog front end (siehe: \ref{wolf}), da der ADC die Schnittstelle der zwei Projekt-Teilbereiche darstellt.\\Es wird versucht die Messwerte vom ADC einzulesen.

\subsection{externer ADC}

\begin{tabular}[h]{|l|l|} %l = left c = center r = right
	\hline
	status & offen\\
	\hline
	beanspruchte Zeit & 8h \\
	\hline
	erledigt bis & 25.03.2021 \\
	\hline
\end{tabular}
Es wird eine Design-Unit geschrieben, die einen Trigger realisiert, welcher später für das Zusammenfassen in Pakete essentiell sein wird. Optional kann dieser für Debugging-Zwecke verwendet werden.

\subsection{PGA}

\begin{tabular}[h]{|l|l|} %l = left c = center r = right
	\hline
	status & offen\\
	\hline
	beanspruchte Zeit & 8h \\
	\hline
	erledigt bis & 01.04.2021 \\
	\hline
\end{tabular}
Der PGA soll mit dem PGA angesteuert werden. Hierfür wird eine eigene Design-Unit geschrieben. Wichtig für die automatische Messbereichsauswahl.

\subsection{automatische Messbereichsauswahl}

\begin{tabular}[h]{|l|l|} %l = left c = center r = right
	\hline
	status & offen\\
	\hline
	beanspruchte Zeit & 8h \\
	\hline
	erledigt bis & 08.04.2021 \\
	\hline
\end{tabular}
Das FPGA-Oszi soll von alleine den geeigneten Messbereich ermitteln können und diesen dann über die PGA's einstellen.

\subsection{kommunikation mit dem PC - user interface}

\begin{tabular}[h]{|l|l|} %l = left c = center r = right
	\hline
	status & offen\\
	\hline
	beanspruchte Zeit & 20h \\
	\hline
	erledigt bis & 29.04.2021 \\
	\hline
\end{tabular}
Die Messdaten sollen in geeigneten Paketen mit Zusatzinformationen, zB.: wie Messbereich, zum user interface (siehe: \ref{radike}) gesendet werden. Die Kommunikation soll in beide Richtungen funktionieren. Extra Chip auf Piggyback vorraussichtlich nötig.

\subsection{digitale Signalverarbeitung}

\begin{tabular}[h]{|l|l|} %l = left c = center r = right
	\hline
	status & offen\\
	\hline
	beanspruchte Zeit & 24h \\
	\hline
	erledigt bis & 20.05.2021 \\
	\hline
\end{tabular}
Es wird versucht den Frequenzgang des analogen Filters vor dem ADC zu kompensieren und optional zusätzlich digitale Filter einzubauen. In dieser Designunit werden zusätzlich die Daten zu Paketen zusammengefasst und für die Weiterleitung zum user  interface (siehe: \ref{radike}) bereit gestellt.


\section{analog front end} \label{wolf}
\section{user interface} \label{radike}
\end{document}
